\documentclass{beamer}

\usepackage{polyglossia}
\usepackage{fontspec}
\usepackage{nameref}
\usepackage{ifthen}

\usefonttheme{professionalfonts}
\usetheme{Antibes}
\useoutertheme{infolines_foot}
\setbeamercovered{transparent=20}

\usepackage[math-style=ISO,vargreek-shape=unicode]{unicode-math}

\setdefaultlanguage[spelling=modern,babelshorthands=true]{russian}
\setotherlanguage{english}

\defaultfontfeatures{Ligatures={TeX}}
\setmainfont{CMU Serif}
\setsansfont{CMU Sans Serif}
\setmonofont{CMU Typewriter Text}
\setmathfont{Latin Modern Math}
\AtBeginDocument{\renewcommand{\setminus}{\mathbin{\backslash}}}

\makeatletter
\newcommand*{\currentname}{\@currentlabelname}
\makeatother
\def\t{\texttt}

\newcommand{\cimg}[2]{%
	\begin{center}%
		\ifthenelse{\equal{#2}{}}{%
			\includegraphics[width=0.75\linewidth]{#1}
		}{%
			\includegraphics[width=#2\linewidth]{#1}
		}%
	\end{center}%
}

\title[3D-сканер]{3D-сканер с Leap Motion}
\author{Гайдашенко Анастасия, Малышева Александра \\ куратор: Кринкин Кирилл Владимирович}
\institute{СПб АУ РАН}
\date{Весна 2016}

\begin{document}

\begin{frame}
	\titlepage
\end{frame}

\section{Цель}

\begin{frame}[t]{Цель проекта}
	Реализовать 3D-сканер с использованием Leap Motion
	
	\cimg{01.png}{0.4}

	3D-сканер — периферийное устройство, анализирующее физический объект и на основе полученных данных создающее его 3D-модель.

\end{frame}

\section{Задачи}
\begin{frame}[t]{Задачи}
	\begin{enumerate}
		\item Изучение API Leap Motion
		\item Поиск оптимального алгоритма Feature Point Detection для данных изображений
		\item Изучение различных алгоритмов  поиска матрицы глубины;\\выбор оптимального
		\item Разработка аппаратной части сканера
		\item Доработка алгоритма построения 3D-модели 
		\item Kinect -> Leap Motion
	\end{enumerate}
\end{frame}

\section{Архитектура проекта}
\begin{frame}[t]{Архитектура проекта}
	\cimg{02.png}{0.6}
\end{frame}

\subsection{Получение данных с Leap Motion}
\begin{frame}[t]{Получение данных с Leap Motion}
	\begin{description}
		\item Задачи
		\begin{enumerate}
			\item Изучение API
			\item Получение изображений с Leap Motion
			\item Проверка полученных результатов
			\item Вывод изображений
		\end{enumerate}
		\item Инструменты
		\begin{enumerate}
			\item Java
			\item Leap Motion
		\end{enumerate}
	\end{description}
\end{frame}

\subsection{Получение Feature Point'ов}
\begin{frame}[t]{Получение Feature Point'ов}
	\begin{description}
		\item Задачи
		\begin{enumerate}
			\item Знакомство с API OpenCV
			\item Изучение алгоритмов семейства Feature Point Detection
			\begin{enumerate}
				\item SurfFeatureDetector
				\item LSDDetector
			\end{enumerate}
			\item Изучение алгоритмов семейства Feature Point Matching
		\end{enumerate}
		\item Инструменты
		\begin{enumerate}
			\item C++
			\item OpenCV
		\end{enumerate}
	\end{description}
\end{frame}

\subsection{Обработка Feature Point'ов}
\begin{frame}[t]{Обработка Feature Point'ов}
	\begin{description}
		\item Задачи
		\begin{enumerate}
			\item Сопоставление Feature Point’ов
			\item Ректификация полученных точек
			\item Расчет координат полученных точек в 3D
			\item Расчет глубины Feature Point’ов
		\end{enumerate}
	\end{description}
	\cimg{06.png}{0.3}
\end{frame}

\subsection{Получение матрицы глубины}
\begin{frame}[t]{Получение матрицы глубины}
	\begin{description}
		\item Задачи
		\begin{enumerate}
			\item Изучение алгоритмов получения матрицы глубины 
			\item Выбор оптимального алгоритма, исходя из специфики изображений
			\item Реализация поиска матрицы глубины по расстоянию до Feature Point’ов
		\end{enumerate}
		\item Инструменты
		\begin{enumerate}
			\item C++
			\item OpenCV
		\end{enumerate}
	\end{description}
\end{frame}

\subsection{Программная часть}
\begin{frame}[t]{Програмная часть}
		Controller -> Исходные данные -> Выбор алгоритма -> Результат 
		\cimg{5.png}{0.5}
\end{frame}

\subsection{Аппаратная часть}
\begin{frame}[t]{Аппаратная часть}
		Состоит из четырех частей: Stepper, Arduino, Каркас, Leap Motion
		\cimg{03.png}{0.5}
\end{frame}

\section{Интеграция}
\begin{frame}[t]{Интеграция}
	\begin{description}
		\item [Первая попытка:] make
		\begin{description}
			\item Возникли проблемы с отладкой
		\end{description}

		\item [Вторая попытка:] .bat
		\begin{description}
			\item В связи с конвертацией изображения из LeapMotion::Image в Java::Image появилась потеря данных
		\end{description}

		\item [Итоговый вариант:] call shell
		\begin{description}
			\item Проблема с потоками
		\end{description}
	\end{description}
\end{frame}

\section{Сложности}
\begin{frame}[t]{Сложности}
	\begin{description}
		\item \textbf{Threads} \newline Корректное последовательное выполнение вызываемых алгоритмов
	
		\item \textbf{API Leap Motion} \newline Отсутствие сериализации у класса LeapMotion::Image
	
		\item \textbf{Feature Point Detection} \newline Среди огромного числа уже существующих алгоритмов выбрать наилучший для LeapMotion::Image

	\end{description}
\end{frame}

\section{Итоги}
\begin{frame}[t]{Итоги}
	\cimg{04.png}{0.9}
\end{frame}

\section{Спасибо за внимание}
\begin{frame}[t]{Спасибо за внимание}
	\begin{description}
		\item[Репозиторий:]  \url{github.com/SashaMalysheva/SPb-AU-LeapMotion}
		\item[Ардуино:]  \url{www.arduino.cc} 
		\item[Leap Motion:] \url{www.leapmotion.com}
	\end{description}
\end{frame}
\end{document}

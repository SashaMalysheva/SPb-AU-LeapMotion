\documentclass{beamer}

\usepackage{polyglossia}
\usepackage{fontspec}
\usepackage{nameref}
\usepackage{ifthen}

\usefonttheme{professionalfonts}
\usetheme{Antibes}
\useoutertheme{infolines_foot}
\setbeamercovered{transparent=20}

\usepackage[math-style=ISO,vargreek-shape=unicode]{unicode-math}

\setdefaultlanguage[spelling=modern,babelshorthands=true]{russian}
\setotherlanguage{english}

\defaultfontfeatures{Ligatures={TeX}}
\setmainfont{CMU Serif}
\setsansfont{CMU Sans Serif}
\setmonofont{CMU Typewriter Text}
\setmathfont{Latin Modern Math}
\AtBeginDocument{\renewcommand{\setminus}{\mathbin{\backslash}}}

\makeatletter
\newcommand*{\currentname}{\@currentlabelname}
\makeatother
\def\t{\texttt}

\newcommand{\cimg}[2]{%
	\begin{center}%
		\ifthenelse{\equal{#2}{}}{%
			\includegraphics[width=0.75\linewidth]{#1}
		}{%
			\includegraphics[width=#2\linewidth]{#1}
		}%
	\end{center}%
}

\title[3D сканер]{3D-сканер с LeapMotion}
\author{Гайдашенко Анастасия, Малышева Александра // куратор: Кринкин Кирилл Владимирович}
\institute{СПб АУ РАН}
\date{Весна 2016}

\begin{document}

\begin{frame}
	\titlepage
\end{frame}

\section{Цель}

\begin{frame}[t]{Цель проекта}
	Реализовать 3D сканер с использование API LeapMotion
	
	\cimg{01.png}{0.45}

\end{frame}

\section{Задачи}
\begin{frame}[t]{Задачи}
	\begin{enumerate}
		\item Изучение API LeapMotion
		\item Изучение и выбор алгоритмов поиска Feature Poin
		\item Изучение и выбор алгоритмов поиска матрицы глубины
		\item Разработка аппаратной части сканера
		\item Доработка алгоритма построения 3D-модели (Kinect -> LeapMotion)
	\end{enumerate}
\end{frame}

\section{Архитектура проекта}
\begin{frame}[t]{Архитектура проекта}
	\cimg{02.png}{0.7}
\end{frame}

\subsection{LeapMotion + Java}
\begin{frame}[t]{LeapMotion + Java}
	\begin{enumerate}
		\item Изучение API LeapMotion, выбор используемых алгоритмов
		\item Получение катинок с LeapMotion, изучение их UI
		\item Проверка результатов, вывод картинок
	\end{enumerate}
\end{frame}

\subsection{OpenCV + C++}
\begin{frame}[t]{OpenCV + C++}
	\begin{enumerate}
	\item знакомство с API OpenCV
	\item изучение алгоритмов семейства Feature Point Detection и Feature Point Matching
	\item
		получение списка feature point’ов:
		\begin{enumerate}
			\item SurfFeatureDetector
			\item LSDDetector
		\end{enumerate}
	\end{enumerate}
\end{frame}

\subsection{LeapMotion + Java}
\begin{frame}[t]{LeapMotion + Java}
	\begin{enumerate}
		\item Сопоставление feature point’ов
		\item Ректификация полученных точек
		\item Получение координат полученных точек в 3D
		\item Расчет глубины feature point’ов
	\end{enumerate}
\end{frame}

\subsection{OpenCV + C++}
\begin{frame}[t]{OpenCV + C++}
	\begin{enumerate}
	\item Выбор лучшего алгоритма, исходя из специфики изображений
	\item Реализация поиска матрицы глубины по расстоянию до feature point’ов
	\end{enumerate}
\end{frame}

\subsection{Интеграция}
\begin{frame}[t]{Интеграция}
	\begin{enumerate}
		\item
		Первая попытка: make
		\begin{enumerate}
			\item Проблема ??
		\end{enumerate}

	\item
		Вторая попытка: .bat
		\begin{enumerate}
			\item Проблема ??
		\end{enumerate}

	\item
		Итоговый вариант: call shell
	\end{enumerate}
\end{frame}

\subsection{Аппаратная часть}
\begin{frame}[t]{Аппаратная часть}
		Состоит из четырех частей: Stepper, Arduino, Каркас, LeapMotion
		\cimg{03.png}{0.6}
\end{frame}

\section{Сложности}
\begin{frame}[t]{Сложности}
	\begin{description}
		\item[API LeapMotion]
			Отсутствие сериализации у класса leap.Image

		\item[Обилие алгоритмов Feature Point Detection]
			Выбор оптимального
			
		\item[Threads]
			Корректное последовательное выполнение вызываемых алгоритмов
	\end{description}
\end{frame}

\section{Итоги}
\begin{frame}[t]{Итоги}
	\cimg{04.png}{0.9}
\end{frame}

\section{Спасибо за внимание}
\begin{frame}[t]{Спасибо за внимание}
	Спасибо за внимание
	\begin{description}
		\item[Репозиторий:]	\url{https://github.com/SashaMalysheva/SPb-AU-LeapMotion}
	\end{description}
\end{frame}
\end{document}
